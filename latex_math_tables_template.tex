\documentclass[a4paper,10pt]{report} % формат бумаги А4, шрифт по умолчанию - 12pt

% заметь, что в квадратных скобках вводятся необязательные аргументы пакетов.
% а в фигурных - обязательные

\usepackage[T2A]{fontenc} % поддержка кириллицы в Латехе
\usepackage[utf8]{inputenc} % включаю кодировку ютф8
\usepackage[english,russian]{babel} % использую русский и английский языки с переносами

\usepackage{indentfirst} % делать отступ в начале параграфа
\usepackage{amsmath} % математические штуковины
\usepackage{mathtools} % еще математические штуковины
\usepackage{mathtext}
\usepackage{multicol} % подключаю мультиколоночность в тексте
\usepackage{graphicx} % пакет для вставки графики, я хз нахуя он нужен в этом документе
\usepackage{listings} % пакет для вставки кода


\usepackage{geometry} % меняю поля страницы

%из параметров ниже понятно, какие части полей страницы меняются:
\geometry{left=2.5cm}
\geometry{right=1cm}
\geometry{top=2cm}
\geometry{bottom=2cm}

\renewcommand{\baselinestretch}{1} % меняю ширину между строками на 1.5
\righthyphenmin=2
\begin{document}

\begin{titlepage}
\newpage

\begin{center}
{\large НАЦИОНАЛЬНЫЙ ИССЛЕДОВАТЕЛЬСКИЙ УНИВЕРСИТЕТ \\
«ВЫСШАЯ ШКОЛА ЭКОНОМИКИ» 							\\
Дисциплина: «Дискретная математика»}

\vfill % заполняет длину страницы вертикально

{\large Домашнее задание 1}

\bigskip

\underline{Исследование комбинационных схем}\\
Вариант 002

\vfill

\begin{flushright}
Выполнил: Абрамов Артем,\\
студент группы БПИ1511\medskip \\
Преподаватель: Авдошин С.М., \\
профессор департамента \\
программной инженерии \\
факультета компьютерных наук
\end{flushright}

\vfill

Москва \number\year

\end{center}
\end{titlepage}

\newpage

\begin{center}
№1.\\
\end{center}


\begin{flushleft}
$18X_7 + 60X_6 + 187X_5 + 57X_4 + 184X_3 + 52X_2 + 253X_1 + 232X_0=244$ \\
Переведём коэффициенты уравнения в двоичную систему счисления.

$18 = 00010010_2$ ,$60 = 00111100_2$ ,$187 = 10111011_2$ ,$57 = 00111001_2$ ,\\
$184 = 10111000_2$ ,$52 = 00110100_2$ ,$253 = 11111101_2$ ,$232 = 11101000_2$ ,$244 = 11110100_2$.\\
Составим расширенную матрицу коэффициентов соответствующей системы линейных уравнений в GF(2) и решим систему. \\

\bigskip
\bigskip

% определяю новую матрицу с чертой
\newenvironment{amatrix}[1]{
	\left(\begin{array}{@{}*{#1}{c}|c@{}}
	}{
	\end{array}\right)
	}
%
\scriptsize{
$	\begin{smallmatrix}
\, & \,
	\end{smallmatrix}
$
$\begin{amatrix}{8}
0&0&1&0&1&0&1&1&1 \\
0&0&0&0&0&0&1&1&1 \\
0&1&1&1&1&1&1&1&1 \\
1&1&1&1&1&1&1&0&1 \\
0&1&1&1&1&0&1&1&0 \\
0&1&0&0&0&1&1&0&1 \\
1&0&1&0&0&0&0&0&0 \\
0&0&1&1&0&0&1&0&0 \\
\end{amatrix}$
$	\begin{matrix}
	(0) \leftrightarrow (3)\\
	(6)\oplus=(0)\\
	\sim
	\end{matrix}
$
$
\begin{amatrix}{8}
1&1&1&1&1&1&1&0&1 \\
0&0&0&0&0&0&1&1&1 \\
0&1&1&1&1&1&1&1&1 \\
0&0&1&0&1&0&1&1&1 \\
0&1&1&1&1&0&1&1&0 \\
0&1&0&0&0&1&1&0&1 \\
0&1&0&1&1&1&1&0&1 \\
0&0&1&1&0&0&1&0&0 \\
\end{amatrix}$
$	\begin{matrix}
	(1) \leftrightarrow (2)\\
	(0)\oplus=(1)\\
	(4)\oplus=(1)\\
	(5)\oplus=(1)\\
	(6)\oplus=(1)\\
	\sim
	\end{matrix}
$
\\
\bigskip
$	\begin{matrix}
	\sim
	\end{matrix}
$
$\begin{amatrix}{8}
1&0&0&0&0&0&0&1&0 \\
0&1&1&1&1&1&1&1&1 \\
0&0&0&0&0&0&1&1&1 \\
0&0&1&0&1&0&1&1&1 \\
0&0&0&0&0&1&0&0&1 \\
0&0&1&1&1&0&0&1&0 \\
0&0&1&0&0&0&0&1&0 \\
0&0&1&1&0&0&1&0&0 \\
\end{amatrix}$
$	\begin{matrix}
	(2) \leftrightarrow (3)\\
	(1)\oplus=(2)\\
	(5)\oplus=(2)\\
	(6)\oplus=(2)\\
	(7)\oplus=(2)\\
	\sim
	\end{matrix}
$
$\begin{amatrix}{8}
1&0&0&0&0&0&0&1&0 \\
0&1&0&1&0&1&0&0&0 \\
0&0&1&0&1&0&1&1&1 \\
0&0&0&0&0&0&1&1&1 \\
0&0&0&0&0&1&0&0&1 \\
0&0&0&1&0&0&1&0&1 \\
0&0&0&0&1&0&1&0&1 \\
0&0&0&1&1&0&0&1&1 \\
\end{amatrix}$
$	\begin{matrix}
	(3) \leftrightarrow (5)\\
	(1)\oplus=(3)\\
	(7)\oplus=(3)\\
	\sim
	\end{matrix}
$
\\
\bigskip
$	\begin{matrix}
	\sim
	\end{matrix}
$
$\begin{amatrix}{8}
1&0&0&0&0&0&0&1&0 \\
0&1&0&0&0&1&1&0&1 \\
0&0&1&0&1&0&1&1&1 \\
0&0&0&1&0&0&1&0&1 \\
0&0&0&0&0&1&0&0&1 \\
0&0&0&0&0&0&1&1&1 \\
0&0&0&0&1&0&1&0&1 \\
0&0&0&0&1&0&1&1&0 \\
\end{amatrix}$
$	\begin{matrix}
	(4) \leftrightarrow (6)\\
	(2)\oplus=(4)\\
	(7)\oplus=(4)\\
	\sim
	\end{matrix}
$
$\begin{amatrix}{8}
1&0&0&0&0&0&0&1&0 \\
0&1&0&0&0&1&1&0&1 \\
0&0&1&0&0&0&0&1&0 \\
0&0&0&1&0&0&1&0&1 \\
0&0&0&0&1&0&1&0&1 \\
0&0&0&0&0&0&1&1&1 \\
0&0&0&0&0&1&0&0&1 \\
0&0&0&0&0&0&0&1&1 \\
\end{amatrix}$
$	\begin{matrix}
	(5) \leftrightarrow (6)\\
	(1)\oplus=(5)\\
	\sim
	\end{matrix}
$
\\
\bigskip
$	\begin{matrix}
	\sim
	\end{matrix}
$
$\begin{amatrix}{8}
1&0&0&0&0&0&0&1&0 \\
0&1&0&0&0&0&1&0&0 \\
0&0&1&0&0&0&0&1&0 \\
0&0&0&1&0&0&1&0&1 \\
0&0&0&0&1&0&1&0&1 \\
0&0&0&0&0&1&0&0&1 \\
0&0&0&0&0&0&1&1&1 \\
0&0&0&0&0&0&0&1&1 \\
\end{amatrix}$
$	\begin{matrix}
	(1)\oplus=(6)\\
	(3)\oplus=(6)\\
	(4)\oplus=(6)\\
	\sim
	\end{matrix}
$
$\begin{amatrix}{8}
1&0&0&0&0&0&0&1&0 \\
0&1&0&0&0&0&0&1&1 \\
0&0&1&0&0&0&0&1&0 \\
0&0&0&1&0&0&0&1&0 \\
0&0&0&0&1&0&0&1&0 \\
0&0&0&0&0&1&0&0&1 \\
0&0&0&0&0&0&1&1&1 \\
0&0&0&0&0&0&0&1&1 \\
\end{amatrix}$
$	\begin{matrix}
	(0)\oplus=(7)\\
	(1)\oplus=(7)\\
	(2)\oplus=(7)\\
	(3)\oplus=(7)\\
	(4)\oplus=(7)\\
	(6)\oplus=(7)\\
	\sim
	\end{matrix}
$
\\
\bigskip
$	\begin{matrix}
	\sim
	\end{matrix}
$
$\begin{amatrix}{8}
1&0&0&0&0&0&0&0&1 \\
0&1&0&0&0&0&0&0&0 \\
0&0&1&0&0&0&0&0&1 \\
0&0&0&1&0&0&0&0&1 \\
0&0&0&0&1&0&0&0&1 \\
0&0&0&0&0&1&0&0&1 \\
0&0&0&0&0&0&1&0&0 \\
0&0&0&0&0&0&0&1&1 \\
\end{amatrix}$

}
\end{flushleft}

\bigskip

\begin{flushleft}
В описаниях преобразований строки обозначены как (1), (2), ..., (8), а выражения $ (i)\oplus{=(j)} $  и $ (i)\leftrightarrow(j) $ соответственно обозначают «заменить все числа в строке ($i$) на их сумму по модулю $2$ с соответствующими числами строки ($j$)» и «поменять строку ($i$) со строкой ($j$) местами». \\
Получаем решение: $X_7$ = 1, $X_6$ = 0, $X_5$ = 1, $X_4$ = 1, $X_3$ = 1, $X_2$ = 1, $X_1$ = 0, $X_0$ = 1. 
\end{flushleft}

\newpage
\begin{flushleft}
Составим таблицу истинности функции F. \\

\bigskip

\begin{tabular}{|c|c|c|c|c|c|c|c|c|}
\hline
A &0 &0 &0 &0 &1 &1 &1 &1 \\
\hline
B &0 &0 &1 &1 &0 &0 &1 &1 \\
\hline
C &0 &1 &0 &1 &0 &1 &0 &1 \\
\hline
F &1 & 0 & 1 & 1 & 1 & 1 & 0 & 1\\
\hline
\end{tabular}
\\
\bigskip

Десятичный номер функции F равен $2^0 + 2^2 + 2^3 + 2^4 + 2^5 + 2^7 = 189$.
\end{flushleft}

\begin{multicols}{2}



\bigskip
\centering
№4.\\
\bigskip
\begin{tabular}{|c|c|c|c|c|c|c|c|c|}
\hline
A &0 &0 &0 &0 &1 &1 &1 &1 \\
\hline
B &0 &0 &1 &1 &0 &0 &1 &1 \\
\hline
C &0 &1 &0 &1 &0 &1 &0 &1 \\
\hline
F  &1 & 0 & 1 & 1 & 1 & 1 & 0 & 1 \\
\hline
$F_A^{\prime}$ &0&1&1&0&0&1&1&0\\
\hline
$F_B^{\prime}$ &0&1&0&1&1&0&1&0\\
\hline
$F_C^{\prime}$ &1&1&0&0&0&0&1&1\\
\hline
$F_{A,B}^{\prime\prime}$ &1&1&1&1&1&1&1&1\\
\hline
$F_{B,C}^{\prime\prime}$ &1&1&1&1&1&1&1&1\\
\hline
$F_{A,C}^{\prime\prime}$ &1&1&1&1&1&1&1&1\\
\hline
$F_{A,B,C}^{\prime\prime\prime}$ &0&0&0&0&0&0&0&0\\

\hline
\end{tabular}\\
\bigskip

№6.\\
\bigskip
\begin{tabular}{|c|c|c|c|c|c|c|c|c|}
\hline
A &0 &0 &0 &0 &1 &1 &1 &1 \\
\hline
B &0 &0 &1 &1 &0 &0 &1 &1 \\
\hline
C &0 &1 &0 &1 &0 &1 &0 &1 \\
\hline
F  &1 & 0 & 1 & 1 & 1 & 1 & 0 & 1 \\
\hline
$F_{(A,B)}^{\prime}$ &1&1&0&0&0&0&1&1\\
\hline
$F_{(B,C)}^{\prime}$ &0&1&1&0&0&1&1&0\\
\hline
$F_{(A,C)}^{\prime}$ &0&1&0&1&1&0&1&0\\
\hline
\end{tabular}
\\
\bigskip
№8.\\
\bigskip
\begin{tabular}{|c|c|c|c|c|c|c|c|c|}
\hline
A &0 &0 &0 &0 &1 &1 &1 &1 \\
\hline
B &0 &0 &1 &1 &0 &0 &1 &1 \\
\hline
C &0 &1 &0 &1 &0 &1 &0 &1 \\
\hline
F &1 & 0 & 1 & 1 & 1 & 1 & 0 & 1 \\
\hline
$F_{(A,B,C)}^{\prime}$ &0&0&0&0&0&0&0&0\\
\hline
\end{tabular}\\
\bigskip
\columnbreak
№5. \\
% РЕШЕНО
\bigskip
$ F_A^{\prime} = B \oplus C $ \\
$ F_B^{\prime} = A \oplus C$ \\
$ F_C^{\prime} = A \equiv B $ \\
$ F_{A,B}^{\prime\prime} =  1$ \\
$ F_{B,C}^{\prime\prime} =  1$ \\
$ F_{A,C}^{\prime\prime} =  1$ \\
$ F_{A,B,C}^{\prime\prime\prime} = 0 $ \\
\bigskip
\bigskip
№7. \\
% Решено
\bigskip
$ F_{(A,B)}^{\prime} = A \equiv B$ \\
$ F_{(B,C)}^{\prime} =  B \oplus C$ \\
$ F_{(A,C)}^{\prime} = A \oplus C$ \\
\bigskip
\bigskip
№9. \\
% Решено
\bigskip
$ F_{(A,B,C)}^{\prime} = 0 $ \\
\end{multicols}

\newpage
\begin{center}
№10.
\end{center}
\begin{flushleft}
$ F(A,B,C)  = 1 \oplus C \oplus AB \oplus BC \oplus AC$ 
\end{flushleft}
\bigskip
\begin{center}
№11.
\end{center}
$ (0,0,0): F(A,B,C) = 1 \oplus C \oplus AB \oplus BC \oplus AC $  \\
$ (0,0,1): F(A,B,C) = A \oplus B \oplus (C\oplus 1) \oplus AB \oplus B(C\oplus 1) \oplus A(C\oplus 1) $ \\
$ (0,1,0): F(A,B,C) = 1 \oplus A \oplus A(B \oplus 1) \oplus (B \oplus 1)C \oplus AC $ \\
$ (0,1,1): F(A,B,C) = 1 \oplus (B \oplus 1) \oplus A(B \oplus 1) \oplus (B \oplus 1)(C\oplus 1) \oplus A(C\oplus 1) $ \\
$ (1,0,0): F(A,B,C) = 1 \oplus B \oplus (A \oplus 1)B \oplus BC \oplus (A \oplus 1)C $ \\
$ (1,0,1): F(A,B,C) = 1 \oplus (A \oplus 1) \oplus (A \oplus 1)B \oplus B(C\oplus 1) \oplus (A \oplus 1)(C\oplus 1)  $ \\
$ (1,1,0): F(A,B,C) = (A \oplus 1) \oplus (B \oplus 1) \oplus C \oplus (A \oplus 1)(B \oplus 1) \oplus (B \oplus 1)C \oplus (A \oplus 1)C $\\
$ (1,1,1): F(A,B,C) = 1 \oplus (C\oplus 1) \oplus (A \oplus 1)(B \oplus 1) \oplus (B \oplus 1)(C\oplus 1) \oplus (A \oplus 1)(C\oplus 1) $ \\
\begin{center} 
№12.  
% Решено
\end{center} 
\begin{flushleft}
$ F(A,B,C) = C \equiv (A + B) \equiv (B + C) \equiv (A + C) $ \\
\end{flushleft}
\begin{center} 
№13.  
% Решено
\end{center} 
\begin{flushleft}
$ (0,0,0): F(A,B,C) = (C\equiv 0) \equiv ((A\equiv 0)+(B\equiv 0)) \equiv ((B\equiv 0)+(C\equiv 0)) \equiv ((A\equiv 0)+(C\equiv 0)) $  \\
$ (0,0,1): F(A,B,C) = 0\equiv (A\equiv 0) \equiv (B\equiv 0) \equiv C \equiv ((A\equiv 0)+(B\equiv 0)) \equiv ((B\equiv 0)+C) \equiv ((A\equiv 0)+C) $ \\
$ (0,1,0): F(A,B,C) = (A\equiv 0) \equiv ((A\equiv 0)+B) \equiv (B+(C\equiv 0)) \equiv ((A\equiv 0)+(C\equiv 0)) $ \\
$ (0,1,1): F(A,B,C) = B \equiv ((A\equiv 0)+B) \equiv (B+C) \equiv ((A\equiv 0)+C) $ \\
$ (1,0,0): F(A,B,C) = (B\equiv 0) \equiv (A+(B\equiv 0)) \equiv ((B\equiv 0)+(C\equiv 0)) \equiv (A+(C\equiv 0)) $ \\
$ (1,0,1): F(A,B,C) = A \equiv (A+(B\equiv 0)) \equiv ((B\equiv 0)+C) \equiv (A+C) $ \\
$ (1,1,0): F(A,B,C) = 0\equiv A \equiv B \equiv (C\equiv 0) \equiv (A+B) \equiv (B+(C\equiv 0)) \equiv (A+(C\equiv 0)) $\\
$ (1,1,1): F(A,B,C) = C \equiv (A+B) \equiv (B+C) \equiv (A+C) $ \\
\end{flushleft}
\begin{center} 
№14.  
% Решено
\end{center}
\begin{flushleft}
$F \notin{T_0} \text{ т.к.}\; F(0,0,0)=1 $ \\
$F \in{T_1} \text{ т.к.}\; F(1,1,1)=1 $ \\
$F \notin{T_\leq} \text{ т.к.}\; (0,0,0) < (0,0,1), \text{ но }\; F(0,0,0)>F(0,0,1) $ \\
$F \notin{T_*} \text{ т.к.}\; F(1,1,1)=F(0,0,0) $ \\
$F \notin{T_L} \text{ т.к.}\; F(0,0,0)= 1 \oplus C \oplus AB \oplus BC \oplus AC $ \\
\end{flushleft}
\begin{center} 
№15.  
\end{center}
\begin{flushleft}
$1 = F(A, A, A) $ \\
$A \equiv B  = F(A, A, B) $ \\
$ AC = F(F(A\equiv1, A\equiv1, C\equiv1), F(A\equiv1, A\equiv1, C\equiv1), 1)= F( F( F(A, A, F(A, A, A)), F(A, A, F(A, A, A)), $ \\ 
$          F(C, C, F(A, A, A))), F(F(A, A, F(A, A, A)), F(A, A, F(A, A, A)), F(C, C, F(A, A, A))), F(A,A,A)) $ \\
$A + B = F(A, B, 1) = F(A, B, F(A, A, A)) $ \\
$B \rightarrow C = F(1, B, C) = F(F(A, A, A), B, C) $ \\
\end{flushleft}
\end{document}
