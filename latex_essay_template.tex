\documentclass[a4paper,12pt]{report} % формат бумаги А4, шрифт по умолчанию - 12pt

% заметь, что в квадратных скобках вводятся необязательные аргументы пакетов.
% а в фигурных - обязательные

\usepackage[T2A]{fontenc} % поддержка кириллицы в Латехе
\usepackage[utf8]{inputenc} % включаю кодировку ютф8
\usepackage[english,russian]{babel} % использую русский и английский языки с переносами

\usepackage{multicol} % подключаю мультиколоночность в тексте, которая обычно нафиг не нужна
\usepackage{graphicx} % пакет для вставки графики, я хз нахуя он нужен в этом документе
\usepackage{listings} % пакет для вставки кода

\usepackage{amsmath} % математические штуковины
\usepackage{mathtools} % еще математические штуковины
\usepackage{mathtext}
\usepackage{verbatim} % пакет для коммнтарий и вставки текста напрямую
\usepackage[parfill]{parskip} % автоматом делает пустые линии между параграфами, там где они есть в тексте

\usepackage{setspace}	 % контроль за размером между строками
\setstretch{1} % ширина между строками 1.2

\usepackage{geometry} % меняю поля страницы
% из параметров ниже понятно, какие части полей страницы меняются:
\geometry{left=2.5cm}
\geometry{right=2cm}
\geometry{top=2cm}
\geometry{bottom=2cm}

\setcounter{secnumdepth}{0} % suppress heading/section/etc numberings

\righthyphenmin=2

\usepackage{indentfirst} % делать отступ в самом первом параграфе, хотя и так ясно что это новый параграф
\setlength\parindent{0pt}	% размер отступа в начале каждого параграфа

\usepackage{csquotes}% превратить " " в красивые двойные кавычки
\MakeOuterQuote{"}		

\begin{document}

\begin{titlepage}

\newpage

\begin{center}

{\large 
НАЦИОНАЛЬНЫЙ ИССЛЕДОВАТЕЛЬСКИЙ УНИВЕРСИТЕТ \\
«ВЫСШАЯ ШКОЛА ЭКОНОМИКИ» 	\\
Дисциплина: «Психология» 	\\
}

\vfill % заполняет длину страницы вертикально

{\large 
	Домашнее задание \\
	Анализ фильма \underline{«Гарольд и Мод (Harold and Maude)»} 	\\
}

\bigskip

\vfill

\begin{flushright}
Выполнил: Абрамов Артем,\\
студент группы БПИ151 \\
%\medskip
факультета компьютерных наук \\
отделения программной инженерии \\
\end{flushright}

\vfill

Москва \number\year

\end{center}
\end{titlepage}

\newpage

%\begin{flushleft}

\begin{center}
{\normalsize
 	Гарольд и Мод (Harold and Maude) \\
 	1971 г. режиссер Хэл Эшби \\
}
\end{center}

\subsubsection{Введение}

Как уже говорилось, в фильме "Гарольд и Мод" можно встретить достаточно много психологических феноменов, я выделил те из них, которые показались мне наиболее интересными. В качестве источников для уточения определений и терминов мной использовались книги:  "Психология" авторы Нуркова В.В., Березанская Н.Б., "Большой психологический словарь" авторы Б. Г. Мещеряков, В. П. Зинченко, "Общая психология" автор  А. Г. Маклаков.


\begin{figure}[!h]
	\centering
	\includegraphics[scale=0.25]{Harold_and_Maude_poster}
	\caption{Постер фильма}
\end{figure}

\newpage


\subsubsection{\underline{Взросление, автономия}}


Поэтому период автономии часто рассматривают как третий этап развития личности, после гетерономии (или конвенциональной морали) и аномии (доморального этапа). С моей точки зрения общение с другими людьми (которого Гарольд был лишен) играет ключевую роль в переходе к автономии. Именно понимание разнообразных подходов к жизненным проблеммам позволяет человеку осознанно выбирать свой путь в мире и мыслить самому за себя.


\subsubsection{\underline{Авторитарность}}

Фактически он берет решение проблеммы в свои руки, отнимая у других людей, какой-либо шанс на инициативу и саморазвитие, таким образом он ставит их в зависимость от себя.


\begin{figure}[!h]
	\centering
	\includegraphics[scale=0.4]{fill_the_form}
	\caption{Миссис Чэйзен отвечает на вопросы вместо Гарольда}
\end{figure}

\newpage

Приведу часть монолога Миссис Чэйзен: \\

{\small
 - Эту анкету нужно заполнить и вернуть. Ты готов, Гарольд? \\
 - Вот первый вопрос, Гарольд. "Трудно ли вам общаться с незнакомыми людьми?" \\
}

\subsubsection{\underline{Рыночный характер (по Эриху Фромму)}}

Интересно что Эрих Фромм исследования проблеммы отчуждения человека от общества и способов преодоления этого отчуждения. С моей точки зрения между матерью и сыном нет любви, они умеют рассуждать только о материальных, а не о духовных ценностях. 

\subsubsection{\underline{Восприятие запахов}}


Интересно что Х. Хеннинг выделял лишь 6 основных типов запахов основываясь на расположении осмофорных групп (химического соединения):  пряный, фруктовый, цветочный, смолистый, гнилостный, пригорелый.

\begin{figure}[!h]
	\centering
	\includegraphics[scale=0.7]{Henning_6_smell_types}
	\caption{Классификация запахов по Хеннингу}
\end{figure}

\newpage

\subsubsection{\underline{5 (или 7) стадий переживания горя}}

В некоторых случаях имеет смысл разбивать стадию отрицания на две фазы: шок, и собственно, отрицание. Также из стадии компромисса можно выделить стадию стыда, когда индивид берет на себя (возможно неоправданно) ответственность за приключившуюся беду.



\subsubsection{Заключение}

В работе выявленны следующие феномены: автономии и взросления, авторитарности, рыночного менталитета, восприятия запахов и 5 (или 7) стадий переживания горя. Более общие феномены, такие как автономия или авторитарность пересекаются между собой на протяжении всего фильма. Наличие одного из них, создает условия для проявления другого. 

В процессе анализа фильма мне удалось узнать много нового.  Например, о важности детского возраста в развитии человека, как это описанно З. Фрейдом. А также и об оппозиции такому  взгляду, например в лице Л. С. Выготского, не могу не процитировать его: "Для Фрейда человек, как каторжник к тачке, прикован к своему прошлому. Вся жизнь определяется в раннем детстве из элементарных комбинаций и вся без остатка сводится к изживанию детских конфликтов". 

\end{document}
